\documentclass[journal,12pt,twocolumn]{beamer}
\usetheme{CambridgeUS}

\usepackage[utf8]{inputenc}
\usepackage{tfrupee}
\usepackage{enumitem}
\usepackage{amsmath}
\usepackage{amssymb}
\usepackage{graphicx}
\usepackage{mathtools}
\providecommand{\sbrak}[1]{\ensuremath{{}\left[#1\right]}}
\providecommand{\lsbrak}[1]{\ensuremath{{}\left[#1\right.}}
\providecommand{\rsbrak}[1]{\ensuremath{{}\left.#1\right]}}
\providecommand{\brak}[1]{\ensuremath{\left(#1\right)}}
\providecommand{\lbrak}[1]{\ensuremath{\left(#1\right.}}
\providecommand{\rbrak}[1]{\ensuremath{\left.#1\right)}}
\providecommand{\cbrak}[1]{\ensuremath{\left\{#1\right\}}}
\providecommand{\lcbrak}[1]{\ensuremath{\left\{#1\right.}}
\providecommand{\rcbrak}[1]{\ensuremath{\left.#1\right\}}}
\newcommand{\myvec}[1]{\ensuremath{\begin{pmatrix}#1\end{pmatrix}}}
\let\vec\mathbf

\title{Assignment 9}
\author{Himanshu Kumar Gupta (AI21BTECH11012)}
\date {May 2022}

\begin{document}
\begin{frame}
 \maketitle   
\end{frame}

\begin{frame}{Contents}
    \tableofcontents
\end{frame}

\section{Question}
\begin{frame}{Question:Ex. 6.62 , Papoulis}
\begin{block}

Suppose X represents the inverse of a chi-square random variable with one degree of freedom, 
and the conditional p.d.f. of Y given X is N(0,x). Show that Y has a Cauchy distribution.
\end{block}
\end{frame}

\section{Solution}
\begin{frame}{Solution}
Let
\begin{align}
    X=1/Z \nonumber
\end{align}
where Z represents a chi-square random variable.So,
\begin{align}
    f_Z\brak{z}&=\frac{z^{-1/2}}{\sqrt{2}\Gamma{\brak{1/2}}}e^{-z/2} 
    =\frac{z^{-1/2}}{\sqrt{2\pi}}e^{-z/2}
\end{align}
\begin{align}
\label{eq:2}
    f_X\brak{x}&=\frac{1}{\lvert \frac{dx}{dz}\rvert} f_Z\brak{1/x}
    =\frac{1}{x^2}\frac{x^{1/2}}{\sqrt{2\pi}}e^{-1/2x}
    =\frac{1}{\sqrt{2\pi}x^{3/2}}e^{-1/2x}\hspace{3mm},\text{$x>0$}
\end{align}
\end{frame}
\begin{frame}
It is also given that
\begin{align}
\label{eq:3}
    f_{Y|X}\brak{y|x}=\frac{1}{\sqrt{2\pi x}}e^{-y^2/2x}
\end{align}
so
    \begin{align}
    \label{eq:4}
        f_{XY}\brak{x,y}&=f_{Y|X}\brak{y|x}f_X\brak{x}\nonumber\\
      &=\frac{1}{\sqrt{2\pi x}}e^{-y^2/2x}\times \frac{1}{\sqrt{2\pi}x^{3/2}}e^{-1/2x} \hspace{6mm}\text{..from  $eq^n$ \eqref{eq:2} and \eqref{eq:3}}   \nonumber\\
      &=\frac{1}{2\pi x^2}e^{-\brak{1+y^2}/2x}
    \end{align}
\end{frame}
\begin{frame}
We know that
\begin{align}
    f_Y\brak{y}&=\int_0^\infty f_{XY}\brak{x,y}dx
    \nonumber\\
    \label{eq:5}
    &=\frac{1}{2\pi}\int_0^\infty \frac{1}{x^2}e^{-\brak{1+y^2}/2x}dx \hspace{15mm}\text{..from \eqref{eq:4}} 
    \end{align}
\end{frame}
\begin{frame}
    Let
    \begin{align}
        u&=\frac{1+y^2}{2x}    \nonumber\\
       du&=\frac{1+y^2}{2}\frac{\brak{-1}}{x^2}   \nonumber
    \end{align}
    putting this in equation \eqref{eq:5}
    \begin{align}
        f_Y\brak{y}=\frac{1}{2\pi}\frac{2}{1+y^2}\int_0^\infty e^{-u}du
        =\frac{1/\pi}{1+y^2}   \hspace{10mm} ,\text{$-\infty<y<\infty$}\nonumber
    \end{align}
    \textbf{Thus Y represents a Cauchy random variable.}
\end{frame}
\end{document}