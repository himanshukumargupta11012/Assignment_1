\documentclass[journal,12pt,twocolumn]{IEEEtran}
\usepackage[utf8]{inputenc}
\usepackage{amsmath}   

\begin{document}

\newcommand{\myvec}[1]{\ensuremath{\begin{pmatrix}#1\end{pmatrix}}}

\let\vec\mathbf


\title{Assignment 1}
\author{\textbf{Himanshu Kumar Gupta(AI21BTECH11012)}}
\maketitle
\date {March 2022}


\textbf{\textit{Problem 3b, ICSE 10 2019:}}\\

 M and N are two points on the X axis and Y axis respectively. 
P (3, 2) divides the line segment MN in the ratio 2 : 3.\\
Find:
\begin{enumerate}
    \item The coordinates of M and N
    \item Slope of the line MN
\end{enumerate}

\textbf{\textit{Solution:}}\\

 let $\vec{P}$ be the position vector of point P\\
  so, $\vec{P}=\myvec{3 \\ 2}$\\
 $\vec{M}$ and $\vec{N}$ be the position vector of point M and N respectively\\
Since M and N are points on x and y axis respectively\\
  So,let $\vec{M}=\myvec{x\\0} and\quad \vec{N}=\myvec{0\\y}$\\
  
  from section formula in vector form, we know that\\
  \begin{equation}
  \label{eq:1}
\vec{P}=\frac{1\times \vec{M} +k\times \vec{N}}{1+k}     
  \end{equation}
 
  where k:1 is ratio in which point P divides the line joining M and N\\
Since P(3,2) divides M and N in ratio 2:3\\
So,$ k=\frac{2}{3}$\\

Now,by applying section formula given in equation \eqref{eq:1} to P on line MN, we get\\
\begin{equation*}
    \vec{P}=\frac{1\times \vec{M}+\frac{2}{3}\times \vec{N}}{1+\frac{2}{3}}
\end{equation*}
\begin{equation*}
    \rightarrow\myvec{3 \\ 2}=\frac{1\times \myvec{x\\0}+\frac{2}{3}\times \myvec{0\\y}}{\frac{5}{3}}
\end{equation*}
\begin{equation*}
\rightarrow\myvec{3 \\ 2}=\frac{3\times \myvec{x\\0}+2\times \myvec{0\\y}}{5}
\end{equation*}
\begin{equation*}
    \rightarrow 5\times \myvec{3\\2}=\myvec{3x\\0}+\myvec{0\\2y}
\end{equation*}
\begin{equation}
\rightarrow\myvec{15\\10}=\myvec{3x\\2y}  \nonumber
\end{equation}
So,
\begin{equation*}
3x=15\quad 2y=10
\end{equation*}
\begin{equation}
\rightarrow x=5\quad \rightarrow y=5       \nonumber
\end{equation}
So,the points M and N would be \textbf{(5,0)} and \textbf{(0,5)} respectively.\\

So,the vector $\vec{MN}=\myvec{0-5\\5-0}$\\

Now,we know that the slope of any vector is

    $\frac{y-component}{x-component}$\\

So,
 \begin{align}
  slope &=\frac{5}{-5}             \nonumber        \\     
  &=\textbf{-1}                    \nonumber
\end{align}

\end{document}
