\documentclass[journal,12pt,twocolumn]{IEEEtran}
\usepackage[utf8]{inputenc}
\usepackage{amsmath}   

\begin{document}


\title{Assignment 1}
\author{\textbf{Himanshu Kumar Gupta(AI21BTECH11012)}}
\date{March 2022}
\maketitle

\textbf{\textit{Problem 3b, ICSE 10 2019:}}\\

 M and N are two points on the X axis and Y axis respectively. 
P (3, 2) divides the line segment MN in the ratio 2 : 3.\\
Find:\\
(i)   the coordinates of M and N\\
(ii)  slope of the line MN.\\

\textbf{\textit{Solution:}}\\

Since M and N are points on x and y axis respectively\\
  So,let M=(x,0) and N=(0,y)\\
  
  from ratio formula for line segment we know that\\
  \begin{equation}
x=\frac{x_1b+x_2a}{a+b}
  \end{equation}
  
  \begin{equation}
  y=\frac{y_1b+y_2a}{a+b}
  \end{equation}
  where a:b is ratio in which (x,y) divides the line joining $(x_1,y_1)$ and $(x_2,y_2)$\\
  
Now,since P(3,2) divides M and N in ratio 2:3\\
So,by applying ratio formula to P on line MN, we get\\
\begin{equation*}
    3=\frac{x*3+0*2}{3+2}
\end{equation*}
\begin{equation*}
    \rightarrow3=\frac{3x}{5}
\end{equation*}
\begin{equation}
    \rightarrow x=5
\end{equation}
\begin{equation*}
    2=\frac{0*3+y*2}{3+2}
\end{equation*}
\begin{equation*}
    \rightarrow2=\frac{2y}{5}
\end{equation*}
\begin{equation}
    \rightarrow y=5
\end{equation}

So,the points M and N would be \textbf{(5,0)} and \textbf{(0,5)} respectively.\\

Now,we know that the slope of any line AB is
 \begin{equation}
 slope=\frac{y_A-y_B}{x_A-x_B}
\end{equation}

So,slope of line MN is
 \begin{equation*}
 slope=\frac{0-5}{5-0}
 \end{equation*}
  \begin{equation}
  \rightarrow slope=\textbf{-1}
  \end{equation}
\end{document}
