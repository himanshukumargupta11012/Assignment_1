\documentclass{beamer}
\usetheme{CambridgeUS}
\usepackage[utf8]{inputenc}
 
\usepackage{tfrupee}
\usepackage{enumitem}
\usepackage{amsmath}
\usepackage{amssymb}
\usepackage{graphicx}
\usepackage{mathtools}
\providecommand{\sbrak}[1]{\ensuremath{{}\left[#1\right]}}
\providecommand{\lsbrak}[1]{\ensuremath{{}\left[#1\right.}}
\providecommand{\rsbrak}[1]{\ensuremath{{}\left.#1\right]}}
\providecommand{\brak}[1]{\ensuremath{\left(#1\right)}}
\providecommand{\lbrak}[1]{\ensuremath{\left(#1\right.}}
\providecommand{\rbrak}[1]{\ensuremath{\left.#1\right)}}
\providecommand{\cbrak}[1]{\ensuremath{\left\{#1\right\}}}
\providecommand{\lcbrak}[1]{\ensuremath{\left\{#1\right.}}
\providecommand{\rcbrak}[1]{\ensuremath{\left.#1\right\}}}

\title{Assignment 5}
\author{\textbf{Himanshu Kumar Gupta (AI21BTECH11012)}}
\date {May 2022}

\begin{document}

\newcommand{\myvec}[1]{\ensuremath{\begin{pmatrix}#1\end{pmatrix}}}
\let\vec\mathbf

\begin{frame}
    \titlepage
\end{frame}
\begin{frame}{Contents}
\tableofcontents
\end{frame}

\section{Question}
\begin{frame}{Example 2.21 , chapter 2 , Papoulis:}
 \textbf{   
Three switches connected in parallel operate independently. Each switch remains closed with probability p.\\
1) Find the probability of receiving an input Signal at the output.\\
2) Find the probability that switch $S_1$ is open given that an input signal is received at the output.}
\end{frame}
\section{Solution($1^{st}$ part)}
\begin{frame}{Solution}
    


Receiving an output signal is random variable because it may be receive or maybe not.

So,
let random variable X denote the situation that output signal is receiving or not, where X=0 means no output signal coming and X=1 means there is output signal coming.

We know that,
\begin{align}
\label{eq:1}
\Pr\brak{X = i} = \frac{n\brak{X = i}}{\sum_{i=0}^1 n\brak{X = i}}
\end{align}
where $i\in\cbrak{{0,1}}$ and $n\brak{X = i}$ is the number of element in event satisfying X=i.
\end{frame}
\begin{frame}
    

Now,

there are only 2 cases for every switch, either it would be open or it would be close.
So,total cases would be $2\times2\times2$

So,
\begin{align}
\label{eq:2}
\sum_{i=0}^1 n\brak{X = i}&=2\times2\times2 \nonumber \\
&=8
\end{align}

X=0 case:

X=0 means we are not receiving input and this can be possible only if  all switch would be open because they are in parallel series.
So,total cases would be only 1.

So,
\begin{align}
\label{eq:3}
n\brak{X = 0}=1  
\end{align}
\end{frame}
\begin{frame}
    
Now,
\begin{align}
\label{eq:4}
    n\brak{X = 1}&=\sum_{i=0}^1 n\brak{X = i}-n\brak{X = 0}  \nonumber\\
    &=8-1    \nonumber \\
    &=7
\end{align}
From equation \eqref{eq:1},probability of getting output,
\begin{align}
\Pr\brak{X = 1} = \frac{n\brak{X = 1}}{\sum_{i=0}^1 n\brak{X = i}}                                  \nonumber
\end{align}
putting values from equations \eqref{eq:2} and \eqref{eq:4},
\begin{align}
\Pr\brak{X = 1}&=\frac{7}{8}         \nonumber\\
&=\textbf{0.875}      \nonumber
\end{align}

\end{frame}
\section{Solution($2^{nd}$ part)}
\begin{frame}
    


 Behavior of switch 1 is also a random variable as it can be close or open.
So,let random variable Y denote the condition of switch 1 where Y=0 means switch is open and Y=1 means switch is close.

Y=0 and X=1 case-

there are only 2 possibility for every switch.We need  the switch 1 to be open.So,only switch 2 and 3 left.So,total cases would be $2\times2$ but there is also a case that we must get output signal.So,we have to remove 1 case in which all switches is open.

So,
\begin{align}
\label{eq:5}
n\brak{Y = 0\cap X=1}&=2\times2-1   \nonumber\\
&=3
\end{align}
\end{frame}
\begin{frame}

We know that,

probability of switch 1 to be open given that an input signal is received at the output is
\begin{align}
\Pr\brak{Y = 0|X=1} = \frac{n\brak{Y = 0\cap X=1}}{n\brak{X=1}}          \nonumber
\end{align}


putting values from equations \eqref{eq:5} and \eqref{eq:4},
\begin{align}
\label{eq:7}
\Pr\brak{Y = 0|X=1}&=\frac{3}{7} \nonumber \\
&=\textbf{.42857}
\end{align}
\end{frame}

\end{document}